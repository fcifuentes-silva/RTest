```\{r setup, include=FALSE\} knitr::opts\_chunk\$set(echo = TRUE)
rmarkdown::pandoc\_convert(``UsingSPARQL\_and\_latex.Rmd'', to =
``latex'', output = ``salida.tex'')

library(SPARQL) library(dplyr)

\begin{verbatim}

## Primer informe con R-Markdown

Este documento lo hemos generado con R-Markdown <http://rmarkdown.rstudio.com>.

Ahora vamos a procesar la consultas que hicimos en el archivo R. Comenzamos por inicializar lo básico a ambas consultas:

```{r comenzar}

# ---- Inicializar datos básicos
endpoint<-"http://datos.bcn.cl/sparql"
prefix <- c('bcnbio','<http://datos.bcn.cl/ontologies/bcn-biographies#>',
            "bio","http://purl.org/vocab/bio/0.1/",
            "foaf","http://xmlns.com/foaf/0.1/",
            "time","http://www.w3.org/2006/time#")
\end{verbatim}

\hypertarget{consulta-1}{%
\subsubsection{Consulta 1}\label{consulta-1}}

Ahora ejecutamos la consulta 1. Para ver las opciones de los code chunk
revisar este enlace \url{https://rmarkdown.rstudio.com/lesson-3.html}.

```\{r consulta1, cache=TRUE \} \# ---- Primera consulta consulta1
\textless- "SELECT count(*) as ?total WHERE \{ ?s a foaf:Person \}"

res1 \textless- SPARQL(url=endpoint, query=consulta1, ns=prefix) res1

\begin{verbatim}
### Consulta 2

Ahora ejecutaremos la segunda consulta y vamos a graficar los datos:

```{r consulta2, cache=TRUE}
consulta2 <- "SELECT ?s ?nombre ?anio WHERE {
     ?s a foaf:Person;
     foaf:name ?nombre;
     bcnbio:hasBorn ?nacimiento .
     ?nacimiento bio:date ?fecha .
     ?fecha time:year ?anio .
     }"

res2 <- SPARQL(url=endpoint,
            query=consulta2,
            ns=prefix)

res2$results %>% summary
\end{verbatim}

\hypertarget{graficar-con-ggplot2}{%
\subsubsection{Graficar con ggplot2}\label{graficar-con-ggplot2}}

Ahora vamos a graficar estos datos en un histograma agregando la función
de densidad de probabilidad \footnote{https://es.wikipedia.org/wiki/Funci\%C3\%B3n\_de\_densidad\_de\_probabilidad}:

```\{r graficar, message=FALSE\} library(ggplot2)

res2\$results \%\textgreater\% ggplot(aes(x=anio)) +
geom\_histogram(aes(y=..density..), colour=``black'', fill=``white'')+
geom\_density(alpha=.2, fill=``\#FF6666'') +
geom\_vline(aes(xintercept=mean(anio)), color=``blue'',
linetype=``dashed'', size=1)

```

\hypertarget{recursos}{%
\subsection{Recursos}\label{recursos}}

Libro español de R \url{https://es.r4ds.hadley.nz/}

Otros libros:

\begin{enumerate}
\def\labelenumi{\arabic{enumi})}
\item
  \url{https://bookdown.org/yihui/rmarkdown/}
\item
  \url{https://bookdown.org/yihui/rmarkdown-cookbook/}
\item
  \url{https://bookdown.org/}
\item
  \url{https://bookdown.org/yihui/rmarkdown-cookbook/word-template.html}
\item
  \url{https://blog.rstudio.com/2020/09/30/rstudio-v1-4-preview-visual-markdown-editing/}
\end{enumerate}

Generar gráficos animados \url{https://gganimate.com/index.html}

Haven (paquete para leer archivos de stata)
\url{https://haven.tidyverse.org}

Libro ggplot2 \url{https://ggplot2-book.org/}

\hypertarget{atajos}{%
\subsection{Atajos}\label{atajos}}

En
\texttt{Tools\ \textgreater{}\ Global\ Options\ \textgreater{}\ Spelling}
es posible elegir el idioma del corrector de texto
